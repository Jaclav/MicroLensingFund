\documentclass{beamer}
\usepackage{amsmath}
\usepackage{amsfonts}
\usepackage{amssymb}
\usepackage{polski}
\usepackage{pgfplots}
\pgfplotsset{compat=1.15}
\usepackage{mathrsfs}
\usetikzlibrary{arrows}
\usetheme{Warsaw}
\usefonttheme[onlymath]{serif}
\usepackage{animate}

\title{Być albo nie być czarną dziurą}
\author[F. Hansdorfer \and J. Winiarczyk \and Ł. Parda
\and T. Gruss]{Franciszek Hansdorfer \and Jacek Winiarczyk \and Łukasz Parda
\and Tomasz Gruss\\{\small Opiekun projektu: dr hab. Radosław Poleski}}
\date{\today}

\begin{document}

\begin{frame}
    \titlepage
\end{frame}

\section{Wprowadzenie do soczewkowania grawitacyjnego}

\subsection{Soczewkowanie grawitacyjne}


\subsection{Mikrosoczewkowanie}
\begin{frame}
    % \[\beta = \theta - \frac{{4GM}}{{c^2}}\frac{{D_d - D_l}}{{D_s D_l}}\frac{1}{\theta}\]

    % \[\theta_E = \sqrt{\frac{{4GM}}{{c^2}}\frac{{D_s-D_l}}{{D_s D_l}}}\]

    % \[u = \frac{{\beta}}{{\theta_E}}\]

    % \[A(u) = \frac{{u^2+2}}{{u\sqrt{{u^2+4}}}}\]
\end{frame}


\subsection{Odchylenia od modelu punktowego źródła i soczewki}

\section{Modelowanie soczewkowania}
\subsubsection{MulensModel}

\section{Rezultaty}

\end{document}