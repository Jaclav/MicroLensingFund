\documentclass{beamer}
\usepackage{amsmath}
\usepackage{amsfonts}
\usepackage{amssymb}
\usepackage{polski}
\usepackage{pgfplots}
\pgfplotsset{compat=1.15}
\usepackage{mathrsfs}
\usetikzlibrary{arrows}
\usetheme{Warsaw}

\title{Być albo nie być czarną dziurą}
\author[F. Hansdorfer \and J. Winiarczyk \and Ł. Parda
\and T. Gruss]{Franciszek Hansdorfer \and Jacek Winiarczyk \and Łukasz Parda
\and Tomasz Gruss\\{\small Opiekun projektu: dr hab. Radosław Poleski}}
\date{\today}

\begin{document}

\begin{frame}
    \titlepage
\end{frame}

\section{Wprowadzenie do soczewkowania grawitacyjnego}

\subsection{Soczewkowanie grawitacyjne}

\subsection{Mikrosoczewkowanie}
\subsection{Odchylenia od modelu punktowego źródła i soczewki}

\section{Modelowanie soczewkowania}
\subsubsection{MulensModel}

\section{Rezultaty}

\end{document}